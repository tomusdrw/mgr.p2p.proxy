%%This is a very basic article template.
%%There is just one section and two subsections.
\documentclass[a4paper,12pt]{scrartcl}
\usepackage[polish]{babel}
\usepackage[utf8]{inputenc}
\usepackage[T1]{fontenc}
\usepackage{datetime}

\usepackage{listings}
\usepackage{graphicx, graphics, epsfig, geometry, pslatex}
\usepackage{amsmath, amssymb}
\usepackage{url}

\usepackage{array}
%\usepackage{times}
\selectlanguage{polish}


\hyphenation{
	po-sta-ci wyż-szy ksią-żce wszy-stkie war-to-ści
	ma-cie-rzy al-go-rytm kt\o-re któ-re
}
\newcommand{\f}{\texttt}
\newcommand{\mytitle}{Distributed caching proxy}
\newcommand{\me}{Tomasz Drwięga}
\newcommand{\s}{ }

\lstset{numbers=left,
	numberstyle=\tiny,
	%basicstyle=\footnotesize,
	showstringspaces=false,
	basicstyle=\footnotesize,
	breaklines=true,
	captionpos=b,
	tabsize=3,
	stepnumber=60,
	firstnumber=1,
	}
%opening
\usepackage{fullpage}
\title{\mytitle}
\author{\me}

\makeindex

\begin{document}

% \parindent0pt
\pagestyle{empty}

\begin{titlepage}
\vspace*{\fill}
\begin{center}
\begin{picture}(300,510)
	\put( 0,520){\makebox(0,0)[l]{\large \bf \textsc{Wydział Podstawowych
	Problemów Techniki}}}
	\put( 0,500){\makebox(0,0)[l]{\large \bf \textsc{Politechniki Wrocławskiej}}}
	\put( 0,280){\makebox(0,0)[l]{\Huge  \bf \textsc{\mytitle}}}
	\put(95,240){\makebox(0,0)[l]{\Large     \textsc{\me}}}
	
	\put(190, 80){\makebox(0,0)[l]{\large  {Praca magisterska napisana}}}
	\put(190, 60){\makebox(0,0)[l]{\large  {pod kierunkiem}}}
	\put(190, 40){\makebox(0,0)[l]{\large  {dra Mirosława Korzeniowskiego}}}
	
	\put(110,-80){\makebox(0,0)[bl]{\large \bf \textsc{Wrocław 2013}}}
\end{picture}
\end{center}
\vspace*{\fill}
\end{titlepage}

\tableofcontents

\newpage

\pagestyle{headings}

\section*{Wstęp}

\subsection{Sformułowanie problemu}

\subsection{Istniejące rozwiązania}
Opis proxy keszujących typu Squid, \cite{piatek}?.


\section{Opis działania}

\section{Analiza bezpieczeństwa i wydajności}
\subsection{System, w którym klucz == url}
Można śledzić co przeglądają użytkownicy.

Nie jest wymagane STORE zasobu - węzły mogą rozpocząć pobieranie podczas operacji SEARCH.

\subsection{System, w którym klucz == hash(url)}
Dodatkowo zawartość plików może być szyfrowana prawdziwym \f{url}.
W ten sposób nie można podglądać zawartości plików. 

\section{Analiza różnych metod cachowania}
Cache wielopoziomowy:
\begin{enumerate}
  \item w pamięci,
  \item na dysku,
  \item w sieci P2P (te dane również w pamięci, na dysku)
\end{enumerate}
Różne strategie przechodzenia między poziomami.


\section{Implementaja}
\subsection{Wybór technologii}
Tutaj próby zrobienia tego w JS jako plugin do przeglądarki.
\subsection{Biblioteki}
Twisted, Entangled.
\subsection{Instalacja i uruchomienie}


\bibliographystyle{plain}

\bibliography{document}

\end{document}

